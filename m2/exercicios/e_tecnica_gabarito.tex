% Options for packages loaded elsewhere
\PassOptionsToPackage{unicode}{hyperref}
\PassOptionsToPackage{hyphens}{url}
\PassOptionsToPackage{dvipsnames,svgnames,x11names}{xcolor}
%
\documentclass[
  letterpaper,
  DIV=11,
  numbers=noendperiod]{scrartcl}

\usepackage{amsmath,amssymb}
\usepackage{lmodern}
\usepackage{iftex}
\ifPDFTeX
  \usepackage[T1]{fontenc}
  \usepackage[utf8]{inputenc}
  \usepackage{textcomp} % provide euro and other symbols
\else % if luatex or xetex
  \usepackage{unicode-math}
  \defaultfontfeatures{Scale=MatchLowercase}
  \defaultfontfeatures[\rmfamily]{Ligatures=TeX,Scale=1}
  \setmainfont[]{Arial}
  \setsansfont[]{Arial}
\fi
% Use upquote if available, for straight quotes in verbatim environments
\IfFileExists{upquote.sty}{\usepackage{upquote}}{}
\IfFileExists{microtype.sty}{% use microtype if available
  \usepackage[]{microtype}
  \UseMicrotypeSet[protrusion]{basicmath} % disable protrusion for tt fonts
}{}
\makeatletter
\@ifundefined{KOMAClassName}{% if non-KOMA class
  \IfFileExists{parskip.sty}{%
    \usepackage{parskip}
  }{% else
    \setlength{\parindent}{0pt}
    \setlength{\parskip}{6pt plus 2pt minus 1pt}}
}{% if KOMA class
  \KOMAoptions{parskip=half}}
\makeatother
\usepackage{xcolor}
\setlength{\emergencystretch}{3em} % prevent overfull lines
\setcounter{secnumdepth}{-\maxdimen} % remove section numbering
% Make \paragraph and \subparagraph free-standing
\ifx\paragraph\undefined\else
  \let\oldparagraph\paragraph
  \renewcommand{\paragraph}[1]{\oldparagraph{#1}\mbox{}}
\fi
\ifx\subparagraph\undefined\else
  \let\oldsubparagraph\subparagraph
  \renewcommand{\subparagraph}[1]{\oldsubparagraph{#1}\mbox{}}
\fi


\providecommand{\tightlist}{%
  \setlength{\itemsep}{0pt}\setlength{\parskip}{0pt}}\usepackage{longtable,booktabs,array}
\usepackage{calc} % for calculating minipage widths
% Correct order of tables after \paragraph or \subparagraph
\usepackage{etoolbox}
\makeatletter
\patchcmd\longtable{\par}{\if@noskipsec\mbox{}\fi\par}{}{}
\makeatother
% Allow footnotes in longtable head/foot
\IfFileExists{footnotehyper.sty}{\usepackage{footnotehyper}}{\usepackage{footnote}}
\makesavenoteenv{longtable}
\usepackage{graphicx}
\makeatletter
\def\maxwidth{\ifdim\Gin@nat@width>\linewidth\linewidth\else\Gin@nat@width\fi}
\def\maxheight{\ifdim\Gin@nat@height>\textheight\textheight\else\Gin@nat@height\fi}
\makeatother
% Scale images if necessary, so that they will not overflow the page
% margins by default, and it is still possible to overwrite the defaults
% using explicit options in \includegraphics[width, height, ...]{}
\setkeys{Gin}{width=\maxwidth,height=\maxheight,keepaspectratio}
% Set default figure placement to htbp
\makeatletter
\def\fps@figure{htbp}
\makeatother

\KOMAoption{captions}{tableheading}
\makeatletter
\makeatother
\makeatletter
\makeatother
\makeatletter
\@ifpackageloaded{caption}{}{\usepackage{caption}}
\AtBeginDocument{%
\ifdefined\contentsname
  \renewcommand*\contentsname{Índice}
\else
  \newcommand\contentsname{Índice}
\fi
\ifdefined\listfigurename
  \renewcommand*\listfigurename{Lista de Figuras}
\else
  \newcommand\listfigurename{Lista de Figuras}
\fi
\ifdefined\listtablename
  \renewcommand*\listtablename{Lista de Tabelas}
\else
  \newcommand\listtablename{Lista de Tabelas}
\fi
\ifdefined\figurename
  \renewcommand*\figurename{Figura}
\else
  \newcommand\figurename{Figura}
\fi
\ifdefined\tablename
  \renewcommand*\tablename{Tabela}
\else
  \newcommand\tablename{Tabela}
\fi
}
\@ifpackageloaded{float}{}{\usepackage{float}}
\floatstyle{ruled}
\@ifundefined{c@chapter}{\newfloat{codelisting}{h}{lop}}{\newfloat{codelisting}{h}{lop}[chapter]}
\floatname{codelisting}{Listagem}
\newcommand*\listoflistings{\listof{codelisting}{Lista de Listagens}}
\makeatother
\makeatletter
\@ifpackageloaded{caption}{}{\usepackage{caption}}
\@ifpackageloaded{subcaption}{}{\usepackage{subcaption}}
\makeatother
\makeatletter
\@ifpackageloaded{tcolorbox}{}{\usepackage[many]{tcolorbox}}
\makeatother
\makeatletter
\@ifundefined{shadecolor}{\definecolor{shadecolor}{rgb}{.97, .97, .97}}
\makeatother
\makeatletter
\makeatother
\ifLuaTeX
\usepackage[bidi=basic]{babel}
\else
\usepackage[bidi=default]{babel}
\fi
\babelprovide[main,import]{portuguese}
% get rid of language-specific shorthands (see #6817):
\let\LanguageShortHands\languageshorthands
\def\languageshorthands#1{}
\ifLuaTeX
  \usepackage{selnolig}  % disable illegal ligatures
\fi
\IfFileExists{bookmark.sty}{\usepackage{bookmark}}{\usepackage{hyperref}}
\IfFileExists{xurl.sty}{\usepackage{xurl}}{} % add URL line breaks if available
\urlstyle{same} % disable monospaced font for URLs
\hypersetup{
  pdftitle={Exercício sobre Técnica de Pesquisa},
  pdfauthor={ABJ},
  pdflang={pt},
  colorlinks=true,
  linkcolor={blue},
  filecolor={Maroon},
  citecolor={Blue},
  urlcolor={Blue},
  pdfcreator={LaTeX via pandoc}}

\title{Exercício sobre Técnica de Pesquisa}
\author{ABJ}
\date{}

\begin{document}
\maketitle
\ifdefined\Shaded\renewenvironment{Shaded}{\begin{tcolorbox}[breakable, sharp corners, interior hidden, borderline west={3pt}{0pt}{shadecolor}, boxrule=0pt, enhanced, frame hidden]}{\end{tcolorbox}}\fi

\hypertarget{orientauxe7uxf5es-gerais}{%
\subsection{Orientações gerais}\label{orientauxe7uxf5es-gerais}}

\begin{itemize}
\tightlist
\item
  Formem grupos de 5 alunos. Todas as questões deverão ser respondidas
  nestes grupos;
\item
  Os grupos terão 2 horas para responder às questões;
\item
  Depois das discussões em grupo, retornaremos à sala para discutir as
  questões conjuntamente;
\item
  Lembrem de anotar as respostas às questões em um documento
  compartilhado entre vocês.
\end{itemize}

\newpage{}

\hypertarget{perguntas}{%
\subsection{Perguntas}\label{perguntas}}

\textbf{Neste exercício, queremos treinar uma habilidade essencial da
realização de pesquisas em jurimetria: a listagem de processos. As
questões a seguir servem para guiar as alunas e os alunos no processo de
listagem.}

\textbf{1) Como um grupo, escolham um das perguntas abaixo para realizar
uma pesquisa.}

\textbf{a) Uma imobiliária quer construir um prédio novo. Ela comprou o
terreno, elaborou o projeto de construção e enviou a documentação para a
Prefeitura. O órgão público, por sua vez, aceitou a documentação e
concedeu o alvará de construção à imobiliária. Entretanto, os moradores
do bairro não querem que esse prédio seja construído e, para tanto, eles
entram com uma ação no TJSP. Assim, vocẽ vai pesquisar as
judicializações que acontecem entre associações de moradores de bairro
contra a concessão, pela Prefeitura, do alvará de construção às
imobiliárias. Vocẽ quer descobrir (i) o que leva os moradores a quererem
barrar a construção de novos edifícios; (ii) o que o Judiciário
responde, isto é, se os pedidos são procedentes ou improcedentes;c.}

\textbf{b) Uma mãe precisa colocar o seu filho na creche para trabalhar,
mas não há vagas disponíveis. Para isso, ela entra com uma ação no TJSP,
buscando adquirir uma vaga na creche. Você vai pesquisar, então, a
judicialização a respeito da fila de espera nas creches de São Paulo.
Você quer descobrir (i) se as crianças que estão pedindo para ``furar
fila'' tem algum motivo especial para tanto; (ii) o que o Judiciário
responde a estes casos; e (iii) o tempo de resposta}

\hypertarget{informauxe7uxf5es-sobre-as-sentenuxe7as}{%
\subsubsection{Informações sobre as
sentenças}\label{informauxe7uxf5es-sobre-as-sentenuxe7as}}

\textbf{2) A partir do tema escolhido, encontre os processos referentes
a ele \href{https://esaj.tjsp.jus.br/cjpg/}{no banco de sentenças, que
chamamos de CJPG (Consulta de Julgados de Primeiro Grau)}, utilizando a
Pesquisa Livre, Classe e Assunto (não é preciso utilizar todos os
campos, apenas aqueles que o grupo julgar necessário). Anotem os termos
pesquisados, as classes e os assuntos e respondam:}

\textbf{a) Todos os processos encontrados estão dentro do escopo da
pesquisa?}

Provavelmente a resposta aqui será não, pois é difícil encontrar
palavras-chave que sejam restritas o suficiente para conter única e
exclusivamente os processos desejados.

\textbf{b) Existem procesoss que estão dentro do escopo, mas que não
foram capturados pela sua pesquisa? Por quê? É possível melhorar?}

Provavelmente sim, pois: (i) a lista de palavras escolhidas pode ter
sido insuficiente; (ii) alguns processos podem estar com classes e
assuntos erradas, ou simplesmente diferentes das que foram escolhidas.

\textbf{c) É possível fazer a pesquisa realizada acima sem utilizar a
Pesquisa Livre, filtrando os processos apenas a partir de Classes e
Assuntos? Por quê? (Essa resposta vai variar entre cada grupo).}

Muito provavelmente não é possível. Isso acontece porque as Classes e
Assuntos não conseguem ser tão específicas quanto queremos para uma
pesquisa. Isso exige, portanto, que utilizemos a Pesquisa Livre para
filtrar melhor o escopo a partir de palavras que existam na sentença.

\hypertarget{informauxe7uxf5es-sobre-os-processos}{%
\subsubsection{Informações sobre os
processos}\label{informauxe7uxf5es-sobre-os-processos}}

\textbf{3) Agora vamos nos perguntar sobre as informações que temos
sobre cada processo.}

\textbf{a) A partir da pesquisa realizada no CJPG na questão 2,
responda: quais são as informações que você teria de cada processo? Isto
é, para cada um dos resultados que a busca gerou, sem acessar as
decisões, quais informações estariam disponíveis?}

\begin{itemize}
\tightlist
\item
  Número do processo
\item
  Classe
\item
  Assunto
\item
  Magistrado
\item
  Comarca
\item
  Foro
\item
  Vara
\item
  Data de disponibilização
\item
  Ementa
\end{itemize}

\textbf{b) Copie o número do processo de um caso que você encontrou na
CJPG e coloque o número deste processo
\href{https://esaj.tjsp.jus.br/cpopg/open.do}{na CPOPG (Consulta de
Processos de Primeiro Grau)}. Quais informações por processo estão
disponíveis neste sistema sem acessar os autos (descreva brevemente)?
Compare essas informações com aquelas encontradas na CJPG. Existem
informações que existiam na CJPG e que não existem na CPOPG? E o
contrário? O que é igual e o que é diferente entre cada sistema, quando
às informações disponíveis?}

\begin{itemize}
\tightlist
\item
  Número do processo
\item
  Classe
\item
  Assunto
\item
  Foro
\item
  Vara
\item
  Juiz
\item
  Data de distribuição
\item
  Código de controle
\item
  Área
\item
  Valor da ação
\item
  Nome das partes
\item
  Função das partes no processo
\item
  Nome do advogado das partes
\item
  Descrição das movimentações processuais
\item
  Data de cada movimentação
\end{itemize}

\textbf{c) É possível responder às perguntas de pesquisa do exercício 1
com as informações disponíveis nas consultas públicas da CJPG e CPOPG?
Indique quais perguntas podem ser respondidas apenas com as informações
e, se houver, quais não podem ser respondidas dessa forma.}

As perguntas sobre se o Judiciário concedeu ou não o pedido e o tempo de
tramitação podem ser respondidas pelas meta-informações, mas exigem um
certo tratamento dos dados. Pelas movimentações processuais, a partir
das Tabelas Processuais Unificadas, podemos identificar algumas
movimentações que indicam, caso tenha tido uma sentença, qual foi o
sentido dessa sentença. Identificando, portanto, a movimentação que se
refere à sentença, podemos pegar a data dessa movimentação e calcular a
diferença de tempo entre ela e a data de distribuição para calcular o
tempo de tramitação.

\textbf{d) Se houver pelo menos uma pergunta que não pôde ser respondida
com as informações disponíveis nos sistemas (CPOPG e CJPG), onde você
poderia encontrar essa informação? Para tanto, indique (i) onde a
informação pode ser acessada, (ii) se existe algum entrave para o acesso
a essa informação e (iii) se é consistente a forma de se encontrar essa
informação ao longo dos processos, isto é, se todos os processos contém
essa informação no mesmo lugar e com a mesma acessibilidade.}

Não é possível encontrar a pergunta sobre o motivo da judicialização.
Para tanto, é necessário acessar a petição inicial (i). A petição deve
ser acessada dentro dos autos de cada processo. Acontece que os autos
não estão disponíveis sem um login de advogado, o que representa um
entrave (ii). Além disso, mesmo com o login de advogado, em alguns
processos pode não ser possível encontrar a petição inicial, uma vez
que, em processos antigos, os autos existem apenas fisicamente ou, se
eles existirem digitalmente, a digitalização dos autos pode ter sido
feita de forma precária, o que faz com que a informação não seja
consistente ao longo dos processos (iii).



\end{document}
